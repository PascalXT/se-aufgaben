\chapter{SQL-Befehle}

Im eigentlichen Informationssystem sind die Queries modular aufgebaut. Das bedeutet, sie greifen auf dieselben Tabellen zu, die zentral erstellt werden. Um das f�r dieses Projekt gew�nschte Verhalten zu erzielen, werden die zentralen Tabellen f�r jedes Query neu erzeugt. Im Folgenden sind f�r jedes Query die SQL statements vermerkt, die zur Berechnung dieses Queries verwendet wurden. Zwischen den Queries gibt es dabei eine �berlappung.

\lstset{language=SQL}

\section{Query 1}
\lstinputlisting{chapters/queries/Q1.sql}

\section{Query 2}
\lstinputlisting{chapters/queries/Q2.sql}

\section{Query 3}
\lstinputlisting{chapters/queries/Q3.sql}

\section{Query 4}
\lstinputlisting{chapters/queries/Q4.sql}

\section{Query 5}
\lstinputlisting{chapters/queries/Q5.sql}

\section{Query 6}
\lstinputlisting{chapters/queries/Q6.sql}

\section{Query 7}
\lstinputlisting{chapters/queries/Q7.sql}

