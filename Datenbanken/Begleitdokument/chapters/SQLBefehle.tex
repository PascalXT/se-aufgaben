\chapter{SQL-Befehle zur Auswertung der Wahl}

In der Implementierung des Wahlinformationssystem sind die Queries modular aufgebaut. Das bedeutet, sie greifen auf dieselben Tabellen zu, die zentral erstellt werden. Um das f�r dieses Projekt gew�nschte Verhalten zu erzielen, werden die zentralen Tabellen f�r jedes Query neu erzeugt. Im Folgenden sind f�r jedes Query die SQL statements vermerkt, die zur Berechnung dieses Queries verwendet wurden. Zwischen den Queries gibt es dabei eine �berlappung, also untertabellen, die in mehreren Queries verwendet werden. Au�erdem sind teilweise Zahlen direkt angegeben, die im Programm als Parameter �bergeben werden k�nnen. Insbesondere gilt dies f�r 2009 und 2005 f�r die Wahljahre.

\lstset{language=SQL}

\section{Query 1 - Sitzverteilung}
Stellt die Sitzverteilung im Bundestag als Tortendiagramm und tabellarisch (Partei, Anzahl der Sitze) dar.
\lstinputlisting{chapters/queries/Q1.sql}

\clearpage
\section{Query 2 - Mitglieder des Bundestages}
Stellt alle Mitglieder des Bundestages als Liste dar.
\lstinputlisting{chapters/queries/Q2.sql}

\clearpage
\section{Query 3 - Wahlkreis�bersicht}
Stellt f�r einen zuf�llig ausgew�hlten Wahlkreis folgende Informationen dar:
\begin{enumerate}
	\item die Wahlbeteiligung
	\item den gew�hlten Direktkandidaten
	\item die prozentuale und absolute Anzahl an Stimmen f�r jede Partei
	\item die Entwicklung der Stimmen im Vergleich zum Vorjahr
\end{enumerate}
\lstinputlisting{chapters/queries/Q3.sql}

\clearpage
\section{Query 4 - Wahlkreissieger}
Stellt die Siegerparteien (Erst-/Zweitstimme) auf einer eingef�rbten Deutschlandkartedar. Alternativ d�rfen Sie auch eine Tabelle verwenden.
\lstinputlisting{chapters/queries/Q4.sql}

\clearpage
\section{Query 5 - �berhangmandate}
Stellt die �berhangmandate pro Partei und Bundesland dar.
\lstinputlisting{chapters/queries/Q5.sql}

\clearpage
\section{Query 6 - Knappste Sieger}
Stellt die Top 10 der knappsten Sieger f�r alle Parteien dar. Die knappstenSieger sind die gew�hlten Erstkandidaten, welche mit dem geringsten Vorsprung gegen�ber ihren Konkurrenten gewonnen haben. Sollte eine Partei keinenWahlkreis gewonnen haben, sollen stattdessen die Wahlkreise ausgegeben werden, in denen
sie am knappsten verloren hat.
\lstinputlisting{chapters/queries/Q6.sql}

\clearpage
\section{Query 7 - Wahlkreis�bersicht (Einzelstimmen)}
Diese Anfrage soll die gleichen Informationen wie Q3 darstellen, aber auf Einzelstimmen durchgef�hrt werden. Die Zufallsauswahl k�nnen Sie auf die ersten 5
bayrischen Wahlkreise (213, ..., 217) beschr�nken.
\lstinputlisting{chapters/queries/Q7.sql}

