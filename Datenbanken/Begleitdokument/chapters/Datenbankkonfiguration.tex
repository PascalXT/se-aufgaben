\chapter{Datenbankkonfiguration}

In diesem Kapitel wird zun�chst detailliert beschrieben wie die Tables des Systems erzeugt werden.
Anschlie�end wird kurz auf die Verwendung von Indizes und Table-Statistiken eingegangen.

\section{SQL-Befehle zur Tabellenerstellung}

Die SQL-Tabellen k�nnen jederzeit neu erzeugt werden durch Verwendung der folgenden Befehle:

\lstset{language=SQL}
\lstinputlisting{chapters/queries/CreateTables.sql}

\section{Indizes und Statistiken}

Um Abfragen zu beschleunigen wurden au�erdem folgende Indizes auf h�ufig verwendete Fremdschl�ssel angelegt:

\lstset{language=SQL}
\lstinputlisting{chapters/queries/CreateIndexes.sql}

Dadurch konnte beispielsweise in einem Testlauf die Laufzeit von Query Q1 von 800ms auf 450ms gesenkt werden.
Nach Importieren der Daten sollte der Statistik-Befehl 

\lstset{language=SQL}
\begin{lstlisting}
RUNSTATS ON TABLE <table> ON ALL COLUMNS ALLOW WRITE ACCESS;
\end{lstlisting}

auf allen Tables ausgef�hrt werden. Dadurch wird die Laufzeit aller Queries nochmal deutlich verbessert.