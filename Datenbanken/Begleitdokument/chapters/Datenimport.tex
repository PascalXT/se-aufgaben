\chapter{Datenimport}

Das System wurde mit Original-Daten der Bundestagswahlen 2009 und 2005 bef�llt. 
Kandidaten, Parteien, Bundesl�nder und Wahlkreise stammen aus vordefinierten Listen. Die Wahlergbnisse (Erst- und Zweitstimmen) stammen von
der offiziellen Webseite des Bundeswahlleiters\footnote{\url{www.bundeswahlleiter.de/de/bundestagswahlen/BTW_BUND_09/veroeffentlichungen/engueltige/kerg.csv }}. Die Daten wurden zun�chst geparst und 
anschlie�end f�r den DB2-Bulkloader in CSV-Dateien geschrieben. Diese wurden dann mittels dem Befehl

\lstset{language=SQL}
\begin{lstlisting}
LOAD FROM <csv-file> OF DEL MODIFIED BY COLDEL; 
METHOD P <columnNumbers> SAVECOUNT 10000 
MESSAGES <logfile> INSERT INTO <table> (<columnString>)
\end{lstlisting}

in die DB2-Datenbank geladen. Alle Import-Methoden sind so geschrieben, dass sie die Tabelle zuerst leert und eventuell aktuell laufende 
Importvorg�nge abbricht. Dadurch k�nnen die Daten auf einfache Weise beliebig oft neu geladen werden. 

Beim Import der Stimmen wird sowohl die Summe der g�ltigen und die Summe der ung�ltigen Stimmen validiert.
Um das sicherzustellen, wird eine passende Kombination von Erst- und Zweitstimme generiert, anstatt jeden Stimmzettel nur einer Partei
zuzuordnen. 
