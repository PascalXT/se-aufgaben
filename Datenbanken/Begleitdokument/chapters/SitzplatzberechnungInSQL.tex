\chapter{Sitzplatzberechnung in SQL}


2009 wurde die Sitzverteilung nach dem Verfahren von Sainte-Lagu�/Schepers ermittelt. Um dieses Verfahren umzusetzen werden drei verschiedene Algorithmen vorgeschlagen, die rechtlich gleichwertig sind. Wir haben uns f�r das H�chstzahlverfahren entschieden. Die erforderlichen Eingabewerte f�r den Algorithmus ermitteln wir durch eine Reihe von tempor�ren Datenbank Tabellen die wir mittels SQL erstellen.

Hinweis zum Verst�ndnis:
im Folgenden wird ``[var] = SELECT ...'' als Platzhalter f�r das Erstellen einer SQL-Variablen im Sinne einer temporary table, view oder WITH table verwendet. [var]  ist dann der Name der tempor�ren Tabelle

Funktion: BerechneSitzverteilungBundestag
Eingabe: Parteien, Bundesl�nder, Wahlkreise, Stimmen, Kandidaten Landeslisten

1. Aggregiere Stimmen

WahlkreisErgebnis1 = SELECT s.KandidatID, s.WahlkreisID, s.Jahr,

   COUNT(id) AS Anzahl

  FROM Stimmen s

  GROUP BY s.KandidatID, s.wahlkreisid;

WahlkreisErgebnis2 = SELECT s.ParteiID, s.WahlkreisID, s.Jahr,

   COUNT(id) AS Anzahl

  FROM Stimmen s

  GROUP BY s.ParteiID, s.wahlkreisid;

ZweitStimmenNachBundesland = SELECT wk.bundeslandid, w2.parteiid,

sum(w2.anzahl) as AnzahlStimmen

    FROM WahlkreisErgebnis2 w2, Wahlkreis wk

    WHERE w2.wahlkreisid = wk.id

    GROUP BY wk.bundeslandid, w2.parteiid;

ZweitStimmenNachPartei = SELECT ParteiID, SUM(AnzahlStimmen)

FROM ZweitStimmenNachBundesland

GROUP BY ParteiID

ZweitStimmenGesamt = SELECT SUM(s.AnzahlStimmen) AS AnzahlStimmen

  FROM ZweitStimmenNachBundesland

2. Direktmandate

Formal:
Direktmandate = $\left\{ k \in Kandidaten | k \mbox{hat in seinem Wahlkreis die meisten Erststimmen erhalten} \right\}$
SQL:
Direktmandate =    
WITH maxErgebnis(WahlkreisId, maxStimmen) as (
SELECT k.dmwahlkreisid, max(w.anzahl)
FROM Wahlergebnis1 v, Kandidat k
WHERE v.kandidatid = k.id
GROUP BY k.dmwahlkreisid)
 
SELECT k.ID, k.ParteiID
FROM maxErgebnis e, Wahlergebnis1 v, Kandidat k
WHERE e.wahlkreisID = v.wahlkreisid
AND e.maxStimmen = v.Anzahl
AND k.ID = v.KandidatID;



3. Parteien im Bundestag

Formal:
ParteienImBundestag = $\left\{p \in Parteien | Hat5Prozent\left(p, ZweitstimmenGesamt\right) \lor
                       Hat3Direktmate\left(p, WahlkreisergebnisseErststimmen\right)\right\}$

SQL:
F�nfProzentParteien =    

SELECT p.ID

FROM Partei p, WahlkreisErgebnis2 v

WHERE v.ParteiID = p.ID

GROUP BY p.ID

HAVING

CAST(SUM(v.Anzahl) AS FLOAT) /

(SELECT AnzahlStimmen FROM ZweitStimmenGesamt)

>= 0.05

DreiDirektmandatParteien =

SELECT dm.ParteiID

FROM Direktmandate dm

GROUP BY dm.ParteiID

HAVING COUNT(*) >= 3;


ParteienImBundestag =

SELECT *

FROM F�nfProzentParteien

UNION

SELECT *

FROM DreiDirektmandatParteien

3. Anzahl Sitze im Bundestag

Formal:
$AnzahlSitze = 598 - \left|\left\{k \in Kandidaten | k \in Direktmandate \land \neg \exists p \in ParteienImBundestag . kandidiertFuer\left(k,p\right)\right\}\right|$
SQL:

WITH
AlleinigeDirektmandate AS
(SELECT dm.ID FROM Direktmandate dm
EXCEPT
SELECT dm.ID FROM Direktmandate dm, ParteienImBundestag p
WHERE p.ID = dm.ParteiID)
SELECT 598 - COUNT(*) FROM AlleinigeDirektmandate


4. Sitzanspruch pro Partei (nach H�chstzahlverfahren)

VerteilungSitzeAufParteien = H�chstzahlverfahren(ZweitStimmenNachPartei, AnzahlSitze)

5. Sitzverteilung nach Landeslisten

For p  Parteien do

VerteilungSitzeAufLandeslisten(p) = H�chstzahlverfahren(

    ZweitStimmenNachBundesland(p), ZweitStimmenNachPartei(p))
End for

6. Sitzverteilung Bundestag

(ENTWURF)

Formal:
Abgeordnete = $\left\{k \in Kandidaten | k \in Direktmandate \lor \exists l \in Landeslisten . Listenplatz\left(k, l\right) < VerteilungSitzeAufLandeslisten\left(k.Partei, l\right) - \left|\left\{k \in Direktmandate | k \mbox{kandidiert in l.Bundesland}\right\}\right| \right\}$

Abgeordnete =

SELECT dm.ID

FROM Direktmandate dm

UNION

SELECT k.ID

FROM Kandiat k

WHERE k.Listenplatz <= (

SELECT t.AnzahlSitze

FROM TempSitze t

WHERE t.ParteiID = k.ParteiID

AND t.BundeslandID = k.BundeslandID

) - (

SELECT COUNT(*)

FROM Direktmandate dm

WHERE dm.ParteiID = k.ParteiID

AND dm.BundeslandID = k.BundeslandID

)

Alternativversion (Problem mit Direkmandatsgewinnern auf Landeslisten):
Abgeordnete =
    SELECT k.ID

FROM Kandidat k, SitzeNachLandeslisten t
    WHERE t.ParteiID = k.ParteiID AND t.BundeslandID = k.BundeslandID
    AND k.Listenplatz - (SELECT COUNT(*)

FROM Direktmandate dm1, Kandidat k1

WHERE dm1.ID = k1.ID

     AND k1.Listenplatz < k.Listenplatz

    AND k1.BundeslandID = k.BundeslandID

    AND k1.ParteiID = k.ParteiID)

<= t.AnzahlSitze -

(SELECT COUNT(*) FROM Direktmandate d2, Kandidat k2 WHERE d2.ID = k2.ID AND k2.ParteiID = k.ParteiID AND k2.BundeslandID = k.BundeslandID)
    UNION

SELECT dm.ID FROM Direktmandate dm

Funktion: H�chstzahlverfahren
Eingabe: StimmenNachPartei, AnzahlSitze
Ausgabe: Sitze

Divisoren = (0.5, ..., 0.5)
Sitze = (0, ..., 0)

RestSitze = MaxSitze
while RestSitze > 0
    SitzKandidaten = { i |  j. StimmenNachPartei[i] / Divisoren[i] <

     StimmenNachPartei[j] / Divisoren[j] }
   
    while |SitzKandidaten| > RestSitze
        SitzKandidaten = SitzKandidaten \ Zuf�lligesElement(SitzKandidaten)
    end while

    for all i in SitzKandidaten do
        Sitze[i] = Sitze[i] + 1
        Divisoren[i] = Divisoren[i] + 1
    end for
end while

     