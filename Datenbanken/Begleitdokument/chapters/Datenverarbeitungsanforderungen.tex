\section{Datenverarbeitungsanforderungen}

Das Wahlsystem soll unter anderem die folgenden Datenverarbeitungsoperationen unterst�tzen. Die nachfolgenden Daten sind Sch�tzungen, die vor der Implementierun erhoben wurden.

\textbf{Prozessbeschreibung:} Berechnung Wahlbeteiligung
\begin{itemize}
	\item H�ufigkeit: j�hrlich
	\item ben�tigte Daten:
		\begin{itemize}
			\item Stimmen
    	\item Wahlbezirke
    	\item Wahlkreise
    	\item Bundesl�nder
		\end{itemize}
	\item Priorit�t: hoch
	\item zu verarbeitende Menge
		\begin{itemize}
			\item 45.000.000 Stimmen
			\item 2500 Wahlbezirke
			\item 299 Wahlkreise
			\item 16 Bundesl�nder
		\end{itemize}  
\end{itemize}

\textbf{Prozessbeschreibung}: Berechnung Wahlergebnis
\begin{itemize}
	\item H�ufigkeit: j�hrlich
	\item ben�tigte Daten:
		\begin{itemize}
			\item Stimmen
			\item Parteien
			\item Kandidaten
			\item Wahlbezirke
			\item Wahlkreise
			\item Landeslisten
		\end{itemize}
	\item Priorit�t: hoch
	\item zu verarbeitende Menge
		\begin{itemize}
			\item 45.000.000 Stimmen
    	\item 30 Parteien
			\item 2.500 Kandidaten
			\item 2.500 Wahlbezirke
			\item 299 Wahlkreise
			\item 500 Landeslisten
		\end{itemize}
\end{itemize}

\textbf{Prozessbeschreibung}: Berechnung Sitzverteilung
\begin{itemize}
	\item H�ufigkeit: j�hrlich
	\item ben�tigte Daten:
		\begin{itemize}
			\item Stimmen
    	\item Parteien
    	\item Kandidaten
    	\item Wahlbezirke
    	\item Wahlkreise
    	\item Landeslisten
    \end{itemize}
	\item Priorit�t: hoch
	\item zu verarbeitende Menge
		\begin{itemize}
			\item 50.000.000 Stimmen
    	\item 30 Parteien
    	\item 2.500 Kandidaten
    	\item 2.500 Wahlbezirke
    	\item 299 Wahlkreise
    	\item 500 Landeslisten
		\end{itemize}
\end{itemize}
