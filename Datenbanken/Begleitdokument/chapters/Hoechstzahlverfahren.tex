\chapter{Details zur Implementierung der Sitzverteilung}
2009 wurde die Sitzverteilung nach dem Verfahren von Sainte-Lagu�/Schepers ermittelt. Um dieses Verfahren umzusetzen werden drei verschiedene Algorithmen vorgeschlagen, die rechtlich gleichwertig sind. F�r die Implementierung in SQL bietet sich das H�chstzahlverfahren an. In Kapitel \ref{chap:SQL-Befehle} ist das Verfahren in folgenden Teiltabellen umgesetzt:

\lstinputlisting{chapters/queries/Hoechstzahlverfahren1.sql}

\lstinputlisting{chapters/queries/Hoechstzahlverfahren2.sql}

Das im H�chstzahlverfahren vorgesehene Schrittweise erh�hen des Divisors f�r jede einzelne Partei wird durch die Divisoren Tabelle verwirklicht. F�r jede Partei wird f�r jeden Divisor eine Punktzahl ermittelt. Anschlie�end werden f�r die h�chsten Punktzahlen Pl�tze im Bundestag vergeben.

Eine der Schwierigkeiten bei der Umsetzung des H�chstzahlverfahren in SQL ist die Behandlung von Stimmgleichst�nden. Wenn zwei oder mehr Parteien exakt gleich viele Stimmen haben, muss der Wahlleiter zuf�llig den Gewinner bestimmen. Die SQL Abfrage hat aber nicht die M�glichkeit w�hrend der Berechnung den Wahlleiter zu kontaktieren. Stattdessen muss der Wahlleiter Zufallsentscheidungen f�r die Maximal m�gliche Anzahl von Entscheidungsf�llen bereits vor der Auswertung bereitstellen. Diese Zufallsentscheidungen sind in der Implementierung zu diesem Dokument in der Tabelle Zufallszahlen abgelegt. Die Tabelle hat die Form (Zeile, Zahl). Die Zeile entspricht der Instanz der Zufallsentscheidung. F�r die 7. m�gliche Zufallsentscheidung wird also die Zeile 7 der Zufallstabelle verwendet. Durch automatische Generierung der Spalte Zeile wird sicher-ge-stellt, dass jede Zeilenzahl zwischen $0$ und $\left(\#Zeilen - 1\right)$ genau einmal vorkommt. Die Zahl wird dann als Entscheidungfinder verwendet: Wenn es n m�gliche Alternativen gibt und die Zufallszahl z die Entscheidung liefern soll, so wird die Alternative $z mod n$ gew�hlt.

Dieses Zufallsverfahren kommt f�r die Ermittlung der Direktmandate, f�r das H�chst\-zahl\-ver\-fah\-ren zur Ermittlung der Sitze pro Partei, sowie f�r das H�chstzahlverfahren zur Ermittlung der Sitze pro Landesliste zum Einsatz. Um Mehrfachverwendung derselben Zufallszahlen zu vermeiden, werden drei getrennte Zufallszahlentabellen verwendet.