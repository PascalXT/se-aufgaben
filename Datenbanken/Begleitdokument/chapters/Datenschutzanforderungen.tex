\chapter{Datenschutzanforderungen}

Das Datenbankmodell erlaubt keinen direkten R�ckschluss auf den W�hler, da dieser nicht im Modell enthalten ist. Allerdings ist es in Einzelf�llen m�glich das Stimmverhalten zu erschlie�en. Dies ist bei einstimmigem Votum innerhalb eines Wahlbezirkes der Fall.

Um die Wahrscheinlichkeit der R�ckverfolgbarkeit zu minimieren sollten folgende Rahmenbedingungen gelten:

\begin{itemize}
    \item Die Wahlbezirke sollten ausreichend gro� sein, so dass einstimmige Wahlen unwahrscheinlich sind.
    \item Es darf keinen Zusammenhang zwischen der ID eines Stimmzettels und dem W�hler geben. Insbesondere darf die ID des Stimmzettels also nicht gleich der ID des W�hlers sein.
    \item Der Vorgang der Stimmabgabe sollte physikalische Mechanismen enthalten um die Anonymit�t der Stimmen zu gew�hrleisten. Insbesondere darf der Stimmzettel w�hrend der Stimmabgabe nicht von au�en einsehbar sein und die Stimmzettel sollten zuf�llig an die W�hler ausgegeben werden um Nachverfolgbarkeit durch markierte Stimmzettel zu verhindern.
    \item Bei der Briefwahl muss der Stimmzettel direkt nach der Verifizierung der Berechtigung von den Daten des W�hlers getrennt werden. In jedem Fall muss der Stimmzettel vor der Sichtung mit anderen Stimmzetteln vermischt werden.
\end{itemize}