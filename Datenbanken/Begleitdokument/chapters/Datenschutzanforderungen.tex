\section{Datenschutzanforderungen}

Um den Datenschutz zu gew�hrleisten wird als ersten Schritt nach Vorlage des Personalausweises ein W�hlpasswort generiert, mit dem der Wahlberechtigte abstimmen kann. Ab dem Zeitpunkt ist seine Stimme entkoppelt von seinem Personalausweis und somit sind R�ckschl�sse nicht mehr direkt m�glich. Indirekte R�ckschl�sse werden dadurch erschwert, dass die Auswertung erst nach Schlie�ung der Wahllokale erfolgt und somit die immer gro�e Stimmzettel-Mengen aggregiert werden.

Zus�tzlich zu den Datentechnischen Anforderungen m�ssen noch mechanische Hindernisse aufgestellt werden. So muss die Stimme im Wahllokal vor Ort abgegeben werden und der Computer darf nicht von au�en einsehbar sein. Das Mitnehmen von Fotoapparaten muss unterbunden werden. Die Hardware des Wahlcomputers muss so gestaltet werden, dass R�ckschl�sse auf die getroffenen Wahlentscheidungen nicht m�glich sind. Zum Beispiel kann der Hauptspeicher klein gew�hlt werden und nach jeder Stimmabgabe gr�ndlich gel�scht werden.