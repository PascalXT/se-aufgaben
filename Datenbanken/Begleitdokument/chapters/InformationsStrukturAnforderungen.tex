\section{Informationsstrukturanforderungen}

Im Folgenden werden die einzelnen Entities anhand ihrer Attribute n�her beschrieben. Die L�ngenangaben sind jeweils in Byte.

\newcommand{\StartTabelle}[2]{
	\textbf{Objektbeschreibung:} #1
	\begin{itemize}
		\item Anzahl: #2
		\item Attribute:
			\begin{itemize}
}

\newcommand{\DefAttribut}[7]{
	\item Attributname: #1
		\begin{itemize}
			\item Typ: #2
			\item L�nge: #3
			\item Wertebereich: #4
			\item Wiederholungen: #5
			\item Definiertheit: #6
			\item Identifizierend: #7
		\end{itemize}
}

\newcommand{\EndeTabelle}{
	\end{itemize}
	\end{itemize}
}


%Objektbeschreibung: T1
%\begin{itemize}
%	\item Anzahl: T2
%	\item Attribute:
%		\begin{itemize}
%			\item Attributname: T3
%			\begin{itemize}
%				\item L�nge: T4
%				\item Wertebereich: T5
%				\item Wiederholungen: T6
%				\item Definiertheit: T7
%				\item Identifizierend: T8
%			\end{itemize}
%		\end{itemize}
%\end{itemize}
%
%\newcommand{\StartTabelle}[2]{
%	\begin{table}[htbp]
%		\centering
%			\begin{tabular}{llllllll}
%		\textbf{Objektbeschreibung:} & #1\\
%		\textbf{Anzahl:} & #2\\
%		\textbf{Attribute:} \\
%		\textbf{Attributname} & \textbf{type} & \textbf{L�nge} & \textbf{Wertebereich} & \textbf{\#Wdh.} & \textbf{Def'heit} & \textbf{Id.} \\
%}
%
%\newcommand{\DefAttribut}[7]{#1 & #2 & #3 & #4 & #5 & #6 & #7 \\}
%
%\newcommand{\EndeTabelle}{
%	\end{tabular}
%	\end{table}
%}

\newcommand{\DefIDTabelle}[2]{
	\StartTabelle{#1}{#2}
		\DefAttribut{ID}{int}{4}{0 ... #2}{0}{100\%}{ja}
	\EndeTabelle
}

\newcommand{\DefShortString}[1]{\DefAttribut{#1}{char}{50}{*}{0}{100\%}{nein}}
\newcommand{\DefString}[2]{\DefAttribut{#1}{char}{#2}{*}{0}{100\%}{nein}}
\newcommand{\DefID}[1]{\DefAttribut{ID}{int}{4}{0 ... #1}{0}{100\%}{ja}}

\DefIDTabelle{Wahlbezirk}{25.000}

\StartTabelle{Wahlkreis}{299}
	\DefID{299}
	\DefShortString{Wahlkreisname}
\EndeTabelle

\StartTabelle{Bundesland}{16}
	\DefID{16}
	\DefShortString{Name}
\EndeTabelle

\StartTabelle{Kandidat}{5000}
	\DefID{5000}
	\DefShortString{Name}
	\DefShortString{Vorname}
	\DefString{Beruf}{255}
	\DefString{Anschrift}{2047}
	\DefAttribut{Geburtsdatum}{date}{4}{*}{0}{100\%}{ja}
	\DefShortString{Geburtsort}
\EndeTabelle

\StartTabelle{Partei}{50}
	\DefID{50}
	\DefShortString{Name}
	\DefShortString{K�rzel}
\EndeTabelle

\StartTabelle{Stimme}{50.000.000}
	\DefID{50.000.000}
	\DefAttribut{Jahr}{int}{4}{1900 ... 2100}{0}{100\%}{nein}
\EndeTabelle

\StartTabelle{Wahlberechtigter}{50.000.000}
	\DefID{50.000.000}
	\DefAttribut{Gew�hlt}{int}{1}{0, 1}{0}{50\%}{nein}
\EndeTabelle

\StartTabelle{ErstStimmenNachWahlkreis}{10.000}
	\DefAttribut{Anzahl}{int}{4}{0 ... 200.000}{0}{100\%}{nein}
	\DefAttribut{Jahr}{int}{4}{1900 ... 2100}{0}{100\%}{nein}
\EndeTabelle

\StartTabelle{ZweitStimmenNachWahlkreis}{10.000}
	\DefAttribut{Anzahl}{int}{4}{0 ... 200.000}{0}{100\%}{nein}
	\DefAttribut{Jahr}{int}{4}{1900 ... 2100}{0}{100\%}{nein}
\EndeTabelle

\StartTabelle{WahlkreisDaten}{10.000}
	\DefID{10.000}
	\DefAttribut{AnzahlWahlberechtigte}{int}{4}{0 ... 200.000}{0}{100\%}{nein}
	\DefAttribut{AnzahlUngueltigeErststimmen}{int}{4}{0 ... 200.000}{0}{100\%}{nein}
	\DefAttribut{AnzahlUngueltigeZweitstimmen}{int}{4}{0 ... 200.000}{0}{100\%}{nein}
	\DefAttribut{Jahr}{int}{4}{1900 ... 2100}{0}{100\%}{nein}
\EndeTabelle

Als Beispiel werden hier zwei Beziehungen n�her beschrieben:

\textbf{Beziehungsbeschreibung}: w�hlt (1. Stimme)
\begin{itemize}
	\item Beteiligte Objekte
		\begin{itemize}
			\item Stimme als W�hler
			\item Kandidat als Empf�nger der Stimme
		\end{itemize}
	\item Anzahl: 45.000.000
\end{itemize}

\textbf{Beziehungsbeschreibung}: w�hlt (2. Stimme)
\begin{itemize}
	\item Beteiligte Objekte
		\begin{itemize}
			\item Stimme als W�hler
			\item Partei als Empf�nger der Stimme
		\end{itemize}
	\item Anzahl: 45.000.000
\end{itemize}