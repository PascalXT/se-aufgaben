\chapter{Vom ER-Modell zum relationalen Schema}

\section{�bersetzung der Entities}

Wahlberechtigter(\underline{ID}, \underline{WahlkreisID}, \underline{WahlbezirkID},  Gew�hlt)\\
Wahlkreis(\underline{ID}, Name)\\
WahlkreisDaten(\underline{ID}, \#Wahlberechtigter, \#Ung�ltigeErststimmen, \#Ung�ltigeZweitstimmen, Jahr)\\
Kandidat(\underline{ID}, Vorname, Nachname, Geburtsdatum, Geburtsort, Anschrift, Beruf)\\
Partei(\underline{ID}, Name, K�rzel)\\
Bundesland(\underline{ID}, Name)\\
Stimme(\underline{ID}, g�ltig, Jahr)\\

\subsection{Spezialbehandlung: schwache Entities}

Der \emph{Wahlbezirk} ist eine schwache Entity und h�ngt von der starken Entity \emph{Wahlkreis} ab:
\\
\\
Wahlbezirk(\underline{WahlkreisID}, \underline{WahlbezirkNr})\\

\subsection{Spezialbehandlung: Generalisierung}

Die Entity \emph{StimmenNachWahlkreis} ist eine Generalisierung von \emph{ErstStimmenNachWahlkreis} und \emph{ZweitStimmenNachWahlkreis}. 
Um Joins zu sparen, haben wir daraus zwei anstatt drei Relationen modelliert:
\\
\\
ErstStimmenNachWahlkreis(\underline{WahlkreisID}, \#Stimmen, Jahr)\\
ZweitStimmenNachWahlkreis(\underline{WahlkreisID}, \#Stimmen, Jahr)\\

\section{Initial-Entwurf Beziehungen}

w�hlt\_in(\underline{WahlberechtigterID}, WahlkreisID)\\
informieren\_�ber(\underline{WahlkreisDatenID}, WahlkreisID)\\
ist\_in(\underline{WahlbezirkID}, WahlkreisID)\\
ist\_abgegeben\_in(\underline{StimmeID}, WahlbezirkID)\\
sind\_abgegeben\_in1(\underline{\#ErstStimmen}, \underline{Jahr}, \underline{WahlkreisID})\\
sind\_abgegeben\_in2(\underline{\#ZweitStimmen}, \underline{Jahr}, \underline{WahlkreisID})\\
liegt\_in(\underline{WahlkreisID}, BundeslandID)\\
kandidiert\_f�r\_Direktmandat(\underline{KandidatID}, WahlkreisID)\\
steht\_auf\_Landesliste(\underline{KandidatID}, BundeslandID, ParteiID, Listenplatz)\\
ist\_Mitglied\_von(\underline{KandidatID}, ParteiID)\\
wurde\_abgegeben\_an1(\underline{\#ErstStimmen}, \underline{Jahr}, \underline{KandidatID})\\
wurde\_abgegeben\_an2(\underline{\#ZweitStimmen}, \underline{Jahr}, \underline{ParteiID})\\
w�hlt1(\underline{StimmeID}, KandidatID)\\
w�hlt2(\underline{StimmeID}, ParteiID)\\

\section{Verfeinerung des Schemas unter Ber�cksichtigung der Beziehungen und Kardinalit�ten}

Wahlberechtigter(\underline{ID}, Gew�hlt, WahlkreisID)\\
Wahlkreis(\underline{ID}, Name, BundeslandID)\\
WahlkreisDaten(\underline{ID}, \#Wahlberechtigter, \#Ung�ltigeErststimmen, \#Ung�ltigeZweitstimmen, Jahr, WahlkreisID)\\
\\
Das Attribut Listenplatz von \emph{steht\_auf\_Landesliste} wird zum Attribut von Kandidat: \\
Kandidat(\underline{ID}, Vorname, Nachname, Geburtsdatum, Geburtsort, Anschrift, Beruf, ParteiID, BundeslandID, WahlkreisID, ParteiID, Listenplatz)\\
Partei(\underline{ID}, Name, K�rzel)\\
Bundesland(\underline{ID}, Name)\\
Wahlbezirk(\underline{WahlkreisID}, \underline{WahlbezirkNr})\\
Stimme(\underline{ID}, g�ltig, Jahr, KandidatID, ParteiID, WahlkreisID, WahlbezirkNr)\\
\\
Zusammen mit ihrem Fremdschl�ssel erhalten die beiden Wahlkreis-aggregierten Stimm-Relationen ihren Schl�ssel:
\\
ErstStimmenNachWahlkreis(\underline{KandidatID}, \underline{\#Stimmen}, \underline{Jahr})\\
ZweitStimmenNachWahlkreis(\underline{ParteiID,} \underline{\#Stimmen}, \underline{Jahr})\\